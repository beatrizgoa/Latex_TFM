\section{Programming language and frameworks}
The  programming language which has been used to develop the thesis is \textit{Python}. \textit{Python} is an object-oriented language and very used nowadays. The version used is the 2.7.\\

\textit{Python} libraries have been used to help to implement the code. The main framework used is \textit{Theano}. \textit{Theano} has been used yo build the convolutional neural network and its training procedure because of the possibility that offers of working with symbolic variables, mathematical expressions and multidimensional arrays. \textit{Theano} documentation could be found in \url{http://deeplearning.net/software/theano/}.\\

\textit{Scikit-learn} is another \textit{Python} library which has been used as the basis of building the classifiers and some of the metrics. Its documentation it is available in the following url \url{https://docs.scipy.org/doc/numpy/}.\\

Others libraries such as \textit{NumPy} or \textit{Matplotlib} have been needed. \textit{NumPy} is a library that allows users to works with multidimensional arrays or random simulations among others qualities. The documentation of this framework is available in \url{https://docs.scipy.org/doc/numpy/}. \textit{Matplotlib} has been utilized as graphic tool to show figures.\\
