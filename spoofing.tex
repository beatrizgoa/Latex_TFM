 %!TEX root = Memoria_TFM.tex
\section{Anti-spoofing and biometrics systems}
At the same time as technology advance, capture systems and processing algorithms have proceeded too. This open an immense quantity of options and one of them is the capability of developing biometrics systems, giving way to identification and recognition systems.\\

A biometric system is a software and hardware system which is based on human physical characteristic or psychological behavior in order to identify a person.\\

\subsection{Historical introduction}
There were in the 1870s when the necessity of identifying people getting physical characteristics appeared. Alphonse Bertillon had the desire of identifying jail prisoners, therefore, to satisfy this need, the skull diameter, arm and foot length were utilized in the USA until the 1920s \cite{Intro_biometrics}.\\

There were in the 1880s when fingerprint and facial identification were proposed. With the appearance of digital signals processing systems in 1960, the voice and the fingerprint biometric systems were begun to be investigated and researchers initiated to think to use this system to people identification in access control security \cite{Intro_biometrics}.\\

Ten years later, the geometry of the hand comenced to be an area of interest for automated identification technologies. The retina and signature verification appeared in the 80s and after a short time the face systems \cite{Intro_biometrics}.\\

The last biometrics systems appeared in the 1990s with the iris recognition \cite{Intro_biometrics}.\\

\subsection{Biometrics}
Biometrics refers to characteristics that human own. These characteristics should be inherent and are exclusive for each human. Because of that, are used for identification.\\

Biometrics could be distinguished as physical or behavioral biometrics. Physical biometrics are characteristics which every human are born with and are genetic, fingerprint, face, DNA, ear and iris are physical biometrics; behavioral biometrics are characteristics which have been matured with person, are  psychological, signature and gait are behavioral biometrics \cite{biometrics_beha}.\\

Nowadays, the quantity of biometrics is considerable. There are not a biometric which is only used or could be selected as optimal, although fingerprint is the most popular biometrics \cite{2d_3d_face}. The characteristic of each biometrics makes it appropriate for each particular application \cite{Intro_biometrics}.\\% The most common biometrics are described \cite{Intro_biometrics}:

%\begin{description}[itemsep=2pt,topsep=8pt,parsep=0pt,partopsep=20pt] \item \textbf{DNA:} using the DNA for recognition is widely used in forensic. This biometric is intrusive and identical twins share the ADN \cite{Intro_biometrics}. \end{description}

%\subsubsection{Face biometric}
%The thesis research is focused on face biometric.\\

\subsection{Biometric system}
A biometric system could be defined as a structure which collects determined biometric data from a user. The input data is processed in order to obtain features with which are compared with template samples. A biometric system could be labeled as a recognition system or verification system depending on the realized task \cite{Intro_biometrics2}:
\begin{description}[itemsep=2pt,topsep=8pt,parsep=0pt,partopsep=20pt]
\item \textbf{Verification system:} user claims an identity and the system validates the identity comparing the acquired data with the stored in the template database. Is a one-to-one comparative.
\item \textbf{Recognition or Identification system:} user data is compared with the all the template database to find user's identity. The identity is not claimed by the user and is given by the system in function of features. Is a one-to-all comparative.
\end{description}

\begin{figure}[htb]
\centering
\includegraphics[width=0.5\textwidth]{images_miscelaneus/verif_identif.PNG}
\caption{Verification and Identification system diagram. Image obtained from \cite{Intro_biometrics2}} \label{fig:Verif_ident}
\end{figure}

Figure \ref{fig:Verif_ident} (figure obtained from \cite{Intro_biometrics2}) represents briefly the process of the acquired data until be compared  with the system database if it is a verification or recognition system. The modules shown in figure \ref{fig:Verif_ident} are described \cite{Intro_biometrics2}:
\begin{description}[itemsep=2pt,topsep=8pt,parsep=0pt,partopsep=20pt]
\item \textbf{Sensor:} the first module is the sensor, which obtains the user's biometric. Depending on the biometrics, the sensor can vary greatly.
\item \textbf{Feature extractor:} input data is processed to obtain the features. The features that are extracted in this module should be the same as the saved in the database.
\item \textbf{Matcher:} features obtained from the previous module are compared or classify with the templates database so the identity of the user could be verified or assigned.
\item \textbf{Database:} the database is formed by templates which are going used as true samples to compare the obtained with sensors. A user should be registered to the system contributing with its biometric template.
\end{description}

Some biometrics could be used, at the same time, in the same biometrics system, adding the biometrics in feature or score level. It is labeled as multimodal and it is used to complement vulnerabilities of biometrics \cite{Spoofing_survey}.

\subsection{Spoofing}
As the same time as biometrics authentication systems appeared, the attacks to trick systems emerge and they are called \textit{spoofing attacks}. For example making plaster molds for geometric hand biometrics or fingerprints would be an attack as the presented in figure \ref{fig:Spoof_fingerprint} (image obtained from \cite{fingerprint_image}).\\

\begin{figure}[htb]
\centering
\includegraphics[width=0.5\textwidth]{images_miscelaneus/spoofing_fingerprint.jpg}
\caption{Spoofing fingerprint. Image obtained from \cite{fingerprint_image}} \label{fig:Spoof_fingerprint}
\end{figure}

Spoofing is referenced to impersonate a biometric system trying to be verified or get an identity when the user is not a genuine. The attack would depend on the biometry and the capturing system \cite{Spoofing_survey}. Attacks are usually made with artificial articles.\\

So as to get the tricks and do not allow spoofing attacks, \textit{anti-spoofing} tries to minimize the attacks or prevent them.\\

Anti-spoofing methods could be distinguished depending on the biometric system module which is combined \cite{Spoofing_survey}:
\begin{description}[itemsep=2pt,topsep=8pt,parsep=0pt,partopsep=20pt]
\item[Sensor level:] extra equipment is added to the sensor, the new device is responsible for getting a living attribute as sweat or blood pressure.
\item[Feature level:] anti-spoofing detection is made after the sample has been acquired. The sample is processed and software decided if it is a genuine or fake user.
\item[Score level:] after the feature extraction, when the scores are obtained, the biometric system decide if the user is genuine or not fusing methods. This method is the newest.
\end{description}

Anti-spoofing systems and scenarios could be evaluated. Depending on how is evaluated two methodologies are distinguished \cite{Spoofing_survey}:
\begin{description}[itemsep=2pt,topsep=8pt,parsep=0pt,partopsep=20pt]
\item[Algorithm-based or technology evaluation:] Evaluate algorithms that prevents anti-spoofing.
\item[System-based or scenario evaluation:] Evaluate the entire acquisition systems including the sensor.
\end{description}

In this project, it is going to be developed the algorithm to detect anti-spoofing attacks when the face is using as biometrics. Given an image, the system has to decide if it is an attack or a genuine user. It would be a verification task.\\

\subsection{Face anti-spoofing}
Face biometrics is very used biometric system because of its attributes since is not intrusive, is easy capturing images and it is considerably accepted by people as biometrics. \\

The principal or main used acquire system are visible light cameras, in spite of other cameras as infra-red or depth camera could be used too. Moreover, using different types of camera could help to detect anti-spoofing attacks.\\

The disadvantages of cameras as sensor are that the illumination need to be controlled and images from different angles could not be suitable, in addition, people expressions or face occlusions are complicated to work with \cite{survey2,2d_3d_face}. On the contrary, the main advantage of using a visible light camera as sensor is the cost, that is very low. \\

The location of verification or identification system could be in a static and specific place as the office entry or could be a wide place as a subway or an airport \cite{survey2}.\\

Attack to this biometric system is also easy to generate, people face images are accessible in social networks. The spoofing attacks could be 3D attacks if 3D masks are used; printed photos, 2D mask and displaying an image or video in a smartphone or tablet are considered 2D attacks \cite{2d_3d_face}. 2D attacks cost cost is not high \cite{distorsion} and lighter that 3D spoofing attacks cost.\\

\begin{figure}[htb]
\centering
\includegraphics[width=0.5\textwidth]{images_miscelaneus/fig_masks.png}
\caption{3D face masks. Image obtained from \cite{3dmask}} \label{fig:3dMasks}
\end{figure}

In figure \ref{fig:3dMasks} (image obtained from \cite{3dmask}) are represented 3D resin faces mask. This masks are used for a 3D face mask database, but are an example of face spoofing attack. 3D mask could be made of silicone too and are more inexpensive than resin mask.\\

2D mask, printed photos or images or videos displayed on a smart device are attacks easier to obtain thanks to the accessibility to personal information in social networks or the easy purchase of visible light cameras.\\

To argue spoofing attacks, feature level is the most studied method. Literature separates into two groups, dynamic and static approach \cite{Spoofing_survey}:
\begin{description}[itemsep=2pt,topsep=8pt,parsep=0pt,partopsep=20pt]
\item[Dynamic:] is based on motion to detect anti-spoofing, for instance, blink eyes or optical flow assessment, therefore a temporal sequence of images are needed. It could be efficient when image attacks are used, but not too accurate when videos are used.
\item[Static:] a unique image is analyzed.
\end{description}

Different techniques, to decide if a user is being impersonated, are being researched and could be used with image and video sequences \cite{Spoofing_survey}.\\

Texture-based methods are one of the most investigated. Local Binary Patterns (LBP),  Histogram of Oriented Gradients (HOG) or Difference of Gaussian (DoG) are algorithms used to extract images features \cite{distorsion,Spoofing_survey}.\\

Motion or liveness detection methods search for movement in videos, asking the user to realize a movement or analyzing the video frames sequence. The liveness could be detected in the blinking eyes, lip movement or head rotation \cite{distorsion,Spoofing_survey}.\\

Others methods, as multispectral-based anti-spoofing, uses images obtained from an alternative to visible light cameras such as infra-red cameras \cite{distorsion}.\\

In general, the explained methods works with 2D face anti-spoofing although 3D anti-spoofing is present too using as a sensor depth cameras \cite{2d_3d_face}. This thesis is focused on 2D face anti-spoofing.\\
