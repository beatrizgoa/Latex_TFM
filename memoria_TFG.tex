%%%%%%%%%%%%%%%%%%%%%%%%%%%%%%%%%%%%%%%%%%%%%%%%%
%%% >>>>>>>>>>> EMPIEZA PREÁMBULO <<<<<<<<<<< %%%
%%%%%%%%%%%%%%%%%%%%%%%%%%%%%%%%%%%%%%%%%%%%%%%%%


\documentclass[12pt,a4paper,titlepage,twoside]{book}
\usepackage[spanish]{babel}	% Idioma en castellano. que ls tablas se llamen tablas
\usepackage{amsmath}		% Proporciona símbolos matemáticos de la American Mathematical Society
\usepackage{amssymb}
\usepackage[utf8]{inputenc}	% Escritura con acentosb
\DeclareUnicodeCharacter{00A0}{ }
\usepackage{latexsym}		% Símbolos
\usepackage{eurosym}       % para el símbolo del euro
\usepackage{fancyhdr}		% Cabecera y pie de página
\usepackage{titlesec}		% Aspecto configurable de los capítulos
\usepackage{longtable}		% Permite las tablas que ocupan más de una página
\usepackage{anysize}		% Configuración de márgenes
\usepackage{color}			% Colores
\usepackage{graphicx}		% Gestión de imágenes
\usepackage{wrapfig}
\graphicspath{ {media/} }	% Carpeta en la que por defecto se encuentran las imagenes
\usepackage{afterpage}
\usepackage{epsfig}			% Entiende EPSs
\usepackage{epstopdf}		% Entiende EPSs
\DeclareGraphicsExtensions{.eps,.ps,.pdf}
\usepackage{hyperref}       %Vínculos
\usepackage{subfigure}		% Subfiguras
\usepackage{minitoc}		% Indice por capítulo #1
\dominitoc					% Indice por capítulo #2
\usepackage{tocbibind}		% Incluir figuras, tablas, referencias e índice a éste
\usepackage{pdfpages}		% Incluir PDFs
\usepackage{anysize}		%Cambiar los margenes

\usepackage{enumitem}		%Para incluir listas no numeradas.
\renewcommand{\labelitemi}{$-$} %Para cambiar los símbolos de una lista no enumerada
\renewcommand{\labelitemii}{$\cdot$}%Para cambiar los símbolos de una lista no enumerada
\usepackage{titlesec}
\usepackage{placeins} %Para Mostrar imagenes almacenadas en cada sección.
%\usepackage{caption} %Cambiar el pie de foto
\usepackage[font={small,it}]{caption}
\usepackage{color} %Colores

%Paquetes para tablas
\usepackage[table,xcdraw]{xcolor}
\usepackage{array} 
\usepackage{booktabs} %cambiar ancho de las líneas


%Cabecera y pie de pagina
%%%%%%%%%%%%%%%%%%%%%%%%%%%%%%%%%%%%%%%%%%
\pagestyle{fancy}
\fancyhf{} 
%\fancyhead[RO]{\leftmark} % En las páginas impares, parte izquierda del encabezado, aparecerá el nombre de capítulo 
\fancyhead[LE]{\leftmark} % En las páginas pares, parte derecha del encabezado, aparecerá el nombre de sección 
\fancyhead[LO,RE]{\thepage} % Números de página en las esquinas de los encabezados 
\renewcommand{\chaptermark}[1]{\markboth{\textbf{\thechapter. #1}}{}} % Formato para el capítulo: N. Nombre 
\renewcommand{\sectionmark}[1]{\markright{\textbf{\thesection. #1}}} % Formato para la sección: N.M. Nombre 
\renewcommand{\headrulewidth}{0.4pt} % Ancho de la línea horizontal bajo el encabezado 
\renewcommand{\footrulewidth}{0.2pt} % Ancho de la línea horizontal sobre el pie 
\setlength{\headheight}{1.5\headheight} % Aumenta la altura del encabezado en una vez y media
\newcommand{\Keywords}[1]{\vfill\noindent{\small{\em Palabras clave}: #1}}
\definecolor{grisclar}{gray}{0.5}
\definecolor{grisfosc}{gray}{0.25}
\newcommand{\myparagraph}[1]{\paragraph{#1}\mbox{}\\}
\setlength{\parindent}{0cm} %Quitamos el indentado de cada parrafo

\makeatletter %Para que al final de cada parte se reanude la numeración de capitulos por 1
\@addtoreset{chapter}{part}
\makeatother



% Modifica algunas traducciones del paquete Babel
%%%%%%%%%%%%%%%%%%%%%%%%%%%%%%%%%%%%%%%%%%%%%%%%%%%
\addto\captionsspanish{
\def\tablename{Tabla}
\def\bibname{Bibliografía}
\def\listtablename{Índice de tablas}
\def\contentsname{Índice general}
\def\mtctitle{Contenidos}
}

% Título, autor y fecha del documento
%%%%%%%%%%%%%%%%%%%%%%%%%%%%%%%%%%%%%%%
\title{\Huge \textbf{``Máster Oficial en Visión Artificial
Tratamiento Digital de Imágenes
Práctica evaluable'}}
\author{\Huge \textit{Beatriz Gómez Ayllón}}



% Referencias internas
%%%%%%%%%%%%%%%%%%%%%%%%
\newcommand{\fullref}[1]{\ref{#1} de la página \pageref{#1}}

% Insertar página en blanco
%%%%%%%%%%%%%%%%%%%%%%%%%%%%%
\newcommand{\blankpage}{
\newpage \thispagestyle{empty}
\emph{  }
\newpage
}


% Elimina la cabecera de la última página vacía del capítulo
%%%%%%%%%%%%%%%%%%%%%%%%%%%%%%%%%%%%%%%%%%%%%%%%%%%%%%%%%%%%%%
\makeatletter
\def\cleardoublepage{\clearpage\if@twoside \ifodd\c@page\else%
   \hbox{}
   \thispagestyle{empty}	% Elimina estilo cabeceras y "plain" enumera la página como la siguiente
   \newpage
   \if@twocolumn\hbox{}\newpage\fi\fi\fi}
\makeatother



%%%%%%%%%%%%%%%%%%%%%%%%%%%%%%%%%%%%%%%%%%%%%%%%%
%%% >>>>>>>>>> FINAL DEL PREÁMBULO <<<<<<<<<< %%%
%%%%%%%%%%%%%%%%%%%%%%%%%%%%%%%%%%%%%%%%%%%%%%%%%




%%%%%%%%%%%%%%%%%%%%%%%%%%%%%%%%%%%%%%%%%%%%%%%%%
%%% >>>>>>>>>>> EMPIEZA LO BUENO <<<<<<<<<<<< %%%
%%%%%%%%%%%%%%%%%%%%%%%%%%%%%%%%%%%%%%%%%%%%%%%%%


\begin{document} 

\mtcaddchapter % Para que el minitock salga bien

\frontmatter
%margen del documento
\marginsize{3.0cm}{2.0cm}{3.5cm}{2.5cm}


%%%%%%%%%%%%%%%% PORTADA %%%%%%%%%%%%%%%%%%
\includepdf[pagecommand={\thispagestyle{empty}}]{media/portada.pdf}

\blankpage

\blankpage % Primera pagina
\blankpage % Primera pagina - reverso


%%%%%%%%%%%%%%%%% ACTA %%%%%%%%%%%%%%%%%%%%%
\includepdf[pagecommand={\thispagestyle{empty}}]{media/Acta.pdf}
\blankpage %reverso del acta



\maketitle
\blankpage %reverso titulo

%%%%%%%%%%%%%%%% DEDICATORIA %%%%%%%%%%%%%%
\input{dedicatoria.tex} 
\blankpage %reverso

%%%%%%%%%%%%%%%% AGRADECIMIENTOS %%%%%%%%%%%%%%%%%%
\chapter*{Agradecimientos}
	\chapter*{Acknowledgements} %Tengo que modificarlo todo porque esta en modo copy-paste
In first place I would like to share my sincere gratitude to the director of this thesis, Dr. Aristeidis Tsitiridismy, for the continuous support, his extensive patience, and the advice that have guide me during the development of the thesis and the research work and have encourage me to improve my work and my professional skills.\\

I am grateful for the help of Dra. Cristina Conde, co-director of this thesis. Furthermore, I would like to thank the FRAV research group due to the given facilities. Particularly, I would like to thank to David Ortega because of his help.\\

I offer my gratitude to the Artificial Vision Master professors who have shared their knowledge and have helped to complete the Master.\\

I would like to share my gratitude to Sara Tascon for her support, effort and English knowledge.

Last but not the least, I would like to thank my family and friends for the patience, cheering me up and for supporting me spiritually throughout the master, the thesis research, writing this document and my life in general.\\

\blankpage %reverso


\chapter*{Resumen}
\input{resumen.tex}
\chapter*{Abstract}
\input{abstract.tex}


\chapter*{Acrónimos}
\input{abreviaciones.tex}
\blankpage % derecha
\blankpage %reverso

%%%%%%%%%%%%%%% ÍNDICE %%%%%%%%%%%%%%%%%%%%%%%
\tableofcontents %Tabla con el Indice

\newpage



%%%%%%%%%%%%%%%%%%%%%%%%%%%%%%%%%%%%%%%%%%%
%%%%%%%%%%%%%%%%% MEMORIA %%%%%%%%%%%%%%%%%
%%%%%%%%%%%%%%%%%%%%%%%%%%%%%%%%%%%%%%%%%%%

\mainmatter
	
\part{Memoria}
\fancyhead[RO]{\bf I MEMORIA} %Encabezado paginas impares con \part
\chapter{Motivación, objetivos e introducción}
		\input{introducion}
\label{cap:intro}
\chapter{Metodología}
	\input{metodologia.tex}
\label{cap:metodologia}
\chapter{Resultados}
	\input{resultados.tex}
\label{cap:resultados}
\chapter{Conclusiones y líneas futuras}
	\input{conclusiones.tex}
\label{cap:conclusiones}


%%%% bibliografía %%%%
\input{Bibliografia.tex}

%%%%%%%%%%%%%%%%%%%%%%%%%%%%%%%%%%%%%%%%%%%
%%%%%%%%%%%%%%%%% PLANOS %%%%%%%%%%%%%%%%%%
%%%%%%%%%%%%%%%%%%%%%%%%%%%%%%%%%%%%%%%%%% %
\fancyhead[RO]{\bf II PLANOS} %Encabezado paginas impares con \part
\part{Planos}
\label{parte:planos}
%Ponemos en blanco las paginas que van a ser sustituidas por planos
\blankpage \blankpage %1º plano
\blankpage \blankpage %2º plano
\blankpage \blankpage %3º plano
\blankpage \blankpage %4º plano
\blankpage \blankpage %5º plano





%%%%%%%%%%%%%%%%%%%%%%%%%%%%%%%%%%%%%%%%%%%
%%%%%%%%% PLIEGO DE CONDICIONES %%%%%%%%%%%
%%%%%%%%%%%%%%%%%%%%%%%%%%%%%%%%%%%%%%%%%%%
\fancyhead[RO]{\bf III PLIEGO Y CONDICIONES} %Encabezado paginas impares con \part

\part{Pliego y Condiciones}
	\input{pliegoCondiciones.tex}


%%%%%%%%%%%%%%%%%%%%%%%%%%%%%%%%%%%%%%%%%%%
%%%%%%%%%%%%%% PRESUPUESTO %%%%%%%%%%%%%%%%
%%%%%%%%%%%%%%%%%%%%%%%%%%%%%%%%%%%%%%%%%%%

\part{Presupuesto}
\fancyhead[RO]{\bf IV PRESUPUESTO} %Encabezado paginas impares con \part
	\input{presupuesto.tex}


%\chapter{Anexos}
\blankpage
\fancyhead[RO]{} %Encabezado paginas impares con \part
\listoffigures 	%indice de imagenes
\blankpage
\listoftables	%indice de tablas
\blankpage
\blankpage
\end{document}