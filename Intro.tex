%!TEX root = Memoria_TFM.tex
\minitoc
\mtcskip

\begin{small}
\emph{In this chapter why this thesis has been realized, its objectives and this document structure are described.\\}
\end{small}
\section{Motivation}
Artificial intelligence is a very interesting field which nowadays is having a lot of importance and it is expanding. Because of my interest I decided studied an Artificial Vision Master and the field that I most like is Machine Learning, more exactly, Deep Learning.\\

How bio-inspired algorithms (as Artificial Neural Networks) learn from the input data and classifying with a magnificent accuracy or how Neural Networks are able to create music and pictures impressed me and is a knowledge that I would like to get.\\

On the other hand, security is a important field that has grown up with technology at the same time that attacks have increase its potential.\\

Human body as an identification key is a fast and private way which could not be lost as easy as a ID card. Reliable biometric identification or recognition  systems have been possible to be implemented with computer advances, also being aware of the attacks that could be produced. Using the acquired knowledge along the Master to prevent attacks or improve security of biometrics systems has cough my interest.\\

The desire of working in a anti-spoofing problem was incremented if the problem was real and I could solve it.\\

The opportunity of working in a security, and more specifically, and anti-spoofing real problem as FRAV group is dealing being part of the \textit{ABC4EU European Project} added with the chance of solving that problem with Deep Learning encourage me to started to work in this thesis.\\

\section{Objectives and contributions}
The principal objective of this thesis is being able to detect face spoofing attacks made on images. Those images are images from genuine users (real users) and images from people trying to impersonate. Those attacks are from different nature, printed images, masks or images shown in devices as smartphones or tablets.\\

To differentiate between real users and attacks images a Convolutional Neural Network (CNN) has been developed using Python language and Theano (a Python library). The secondary objective is learning this artificial intelligence research field that nowadays has a significance importance and obtain an CNN architecture which is able to learn the spoofing features which optimise the classification task.\\

Images from different database are feed to the CNN and the output features obtained are the input of the classifiers. Various classifiers has been trained (KNN, logistic regression, Support Vector Machine, ...) for this purpose. Next goal is obtaining the best classifier that suits better to this anti-spoofing detection problem and if dimensionality reduction algorithms as LDA or PCA are needed.\\

The main contribution of this thesis would be the best combination between a specific CNN architecture and the classifier to classify rightly the most number of samples.\\

Trying to differentiate characteristics of the rightly classified samples and incorrectly classified samples would be interesting. The following objective is identify a pattern (if exists) of incorrectly classified samples.\\

Because of the fact that some databases are going to be used for each experiment, the last objective is presented as what database performs a better classification task and why.\\

\section{Thesis structure}
This thesis is formed by a brief introduction, presented in chapter \ref{ch:introduction}, where the motivation to carry out this project has been exposed and this thesis main objectives are described.\\

Before go ahead to the development of the main part, in chapter \ref{theorics}, a short introduction to neural network is presented as well as anti-spoofing projects developed in others research articles. The classifiers and metric used along the project is shown in this chapter too.\\

In chapter \ref{ch:methodology}, the main part is developed, here the steps that have led to the final architecture of the neural networks are explained.\\

In the following chapter \ref{ch:results} the results obtained in the previous chapter \ref{ch:methodology} are presented and discussed.\\

The last chapter \ref{ch:concl}, the conclusions obtained along the thesis are discussed and the future work is proposed.\\
