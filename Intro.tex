%!TEX root = Memoria_TFM.tex
\minitoc
\mtcskip

\begin{small}
\emph{In this chapter, the reasons that have inspired to effort in developing this work are exposed. The thesis objectives and this document structure are described as well.\\}
\end{small}
\section{Motivation}
Artificial intelligence (AI) is a very researched field which nowadays is having a lot of importance and it is expanding. Because of my interest on AI,  I decided to study an Artificial Vision Master and the most that I would like to investigate is Machine Learning, more exactly, Deep Learning.\\

How bio-inspired algorithms (as Artificial Neural Networks) learn from the input data and classify with a magnificent accuracy the test samples or how Neural Networks are able to create music and pictures impressed me and is a knowledge that I would like to get.\\

On the other hand, security is an important area that has grown up with technology at the same time that attacks have increase its potential. Helping society positively working in this field would be gratifying.\\

Human body as an identification key is a fast and private way which could not be lost as easy as a ID card. Reliable biometric identification or recognition  systems have been possible to be implemented with computer advances, also being aware of the attacks that could be produced. Using the acquired knowledge along the Master to prevent attacks or improve security of biometrics systems has cough my interest.\\

The desire of working in a anti-spoofing problem was incremented if the solution to the problem could be implemented in a real scenario or I could contribute with my knowledge to the investigation helping others researchers.\\

The opportunity of working in a security, and more specifically, and anti-spoofing real problem as FRAV group is dealing being part of the \textit{ABC4EU European Project} added with the chance of solving that problem with Deep Learning encourage me to started to work in this thesis.\\

\section{Objectives and contributions}
The principal objective of this thesis is being able to detect face spoofing attacks made on images. Those images are from genuine users (real users) and images from people trying to impersonate (attacks). Attacks could be from different nature, printed images, masks or images shown in devices as smartphones or tablets.\\

To differentiate between real users and attacks images a Convolutional Neural Network (CNN) has been developed using Python language programming and Theano (a Python library). The secondary objective is learning this artificial intelligence research field that nowadays has a significance importance obtaining a CNN architecture which is able to learn the spoofing features in order to optimize the classification task.\\

Images from different database are feed to the CNN and the output features obtained are the input of the classifiers. Various classifiers has been trained (KNN, logistic regression, Support Vector Machine, ...) for this purpose. Next goal is obtaining the best classifier that suits better to this anti-spoofing detection problem and if dimensionality reduction algorithms as LDA or PCA are needed.\\

The main contribution of this thesis would be the best combination between a specific CNN architecture and the classifier to classify rightly the most number of samples.\\

Trying to differentiate characteristics of the rightly classified samples and incorrectly classified samples would be interesting. The following objective is identify a pattern (if exists) of incorrectly classified samples.\\

Because of the fact that some databases are going to be used for each experiment, the last presented objective would be  selecting a database whose performance is the best with the classification task and why.\\

\section{Thesis structure}
This thesis is formed by a brief introduction, presented in chapter \ref{ch:introduction}, where the motivation to carry out this project has been exposed and this thesis main objectives are described.\\

With the purpose of knowing the advances in this thesis area, a literature review is detailed in chapter \ref{ch:review}.

Before go ahead to the development of the main part, in chapter \ref{theorics}, a short introduction to neural network is presented as well as anti-spoofing definitions.

In chapter \ref{ch:methodology}, the methodology is presented. In this chapter the language programming, the databases, classifiers and metrics used in the document are detailed.\\

Experiments that have been realized are specified in chapter \ref{ch:experiments}.\\

In chapter \ref{ch:results}, the results obtained in previous chapter \ref{ch:methodology} are presented and discussed.\\

In the last chapter \ref{ch:concl}, the conclusions obtained along the thesis are explained and the future work as well\\
