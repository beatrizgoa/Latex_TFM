%!TEX root = Memoria_TFM.tex
\minitoc
\mtcskip

\begin{small}
\emph{In this chapter, the reasons that have inspired to effort in developing this work are exposed. The thesis objectives and this document structure are described as well.\\}
\end{small}
\section{Motivation}
Artificial intelligence (AI) is a very researched field which nowadays is having a lot of importance and it is expanding. Because of my interest in AI,  I decided to study an Artificial Vision Master and the most that I would like to investigate is Machine Learning, more exactly, Deep Learning.\\

How bio-inspired algorithms (as Artificial Neural Networks) learn from the input data and classify with a magnificent accuracy the test samples or how Neural Networks are able to create music and pictures impressed me and is a knowledge that I would like to get.\\

The Human body as an identification key is fast, private identification and recognition systems. Biometrics is a technology that is currently being implemented in society and its presence is being increasing. At the same time as biometric systems increase its potential, attacks are more powerful. From the acquired knowledge along the Master, I would like to expand them to be able to contribute preventing attacks or improving the security of biometrics systems. Applied biometrics has caught my interest. \\

Connect this two technologies, biometrics and Deep learning, is an opportunity to deepen both interesting fields. The Anti-spoofing with CNN researched works has started a few years ago. Developing this thesis could contribute with investigation.\\

FRAV group are currently working on \textit{ABC4EU European Project}, where face biometric is utilized. Learning from a project that is currently being developed with technology that interested me, is a generous opportunity. And developing a thesis with them and with biometrics and Deep Learning is an excellent chance that has encouraged me.\\

\section{Objectives and contributions}
\subsubsection{Objectives}
The principal objective of this thesis is being able to detect face spoofing attacks made on images with Deep Learning. These images are from genuine users (real users) and images from people trying to impersonate (attacks) with printed images, masks or images shown in devices as smartphones or tablets.\\

the second objective is obtaining basic theoretical Deep Learning fundamentals knowledge as the same way as learning Theano, the main Deep Learning Python library which is going to be used.\\

Next objective is to build a Convolutional Neural Network architecture which is adapted for this thesis and would use to extract the feature of databases images.\\

Following objective is working with different classifiers and dimensionality reduction algorithms and selecting the best one from the results obtained.\\

Next objective is selecting the most correctly database from the compared and analyzed results.\\

Next objective is finding a pattern among the incorrectly classified samples, in order to know if physical attributes contribute the bad classification.\\

\subsubsection{Contributions}
This thesis contribute with a Convolutional Neural Network architecture and a classifier which optimize the classification.\\

Also, a study among the different used classifiers and dimensionality reduction algorithms is presented for each database.\\

The next contribution is the argument among the results of each database with the selection of the database whose samples are the most correctly classified.\\

\section{Thesis structure}
This thesis is constituted by a brief introduction, presented in chapter \ref{ch:introduction}, where the motivation to carry out this project has been exposed and this thesis main objectives are described.\\

With the purpose of knowing the advances in this thesis area, a literature review is detailed in chapter \ref{ch:review}.\\

Before go ahead to the development of the main part, in chapter \ref{theorics}, a short introduction to neural network is presented as well as anti-spoofing definitions.\\

In chapter \ref{ch:methodology}, the methodology is presented. In this chapter the language programming, the databases, classifiers and metrics used in the document are detailed.\\

Experiments that have been realized are specified in chapter \ref{ch:experiments}.\\

In chapter \ref{ch:results}, the results obtained in previous chapter \ref{ch:methodology} are presented and discussed.\\

In the last chapter \ref{ch:concl}, the conclusions obtained along the thesis are explained and the future work as well\\
