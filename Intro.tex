%!TEX root = Memoria_TFM.tex
%\minitoc
%\mtcskip

\begin{small}
\emph{In this chapter, the reasons that have led to developing this work are exposed. Furthermore, the objectives of this thesis and the structure of the document are also described.}
\end{small}
\section{Motivation}
Artificial intelligence (AI) is a remarkably researched field which in recent years is gaining importance and is rapidly expanding. Due to my interest in AI, I decided to study an Artificial Vision Master having as a main motivation learning about  Machine Learning, more specifically, Deep Learning.\\

How bio-inspired algorithms (as Artificial Neural Networks) learn from the input data and classify with a magnificent accuracy the testing samples or how Neural Networks are able to create music and pictures impressed me and it was a field of learning that I wished to discover in more depth.\\

The Human body as means of identification, hence as a key, is a fast and private identification and recognition system. Biometrics is a technology that is currently being implemented in our society and its presence is increasing. At the same time that biometric systems increase its potential, attacks are more and more powerful. From the acquired knowledge throughout the Master, I would like to expand them to be able to contribute preventing attacks or improving the security of biometrics systems. Applied biometrics has inspired my interest.\\

Connecting these two technologies, biometrics and Deep learning, is an opportunity to deepen both interesting fields. The Anti-spoofing with CNN researched works began a few years ago. Developing this thesis could contribute to this investigation.\\

FRAV group is currently working on the \textit{ABC4EU} European Project, where face biometric is utilized. Learning from a project that is currently being developed with technology that interested me, was a favorable and beneficent opportunity. Whatsmore, developing a thesis with them and with biometrics and Deep Learning has been an excellent opportunity that has greatly encouraged me.

\section{Objectives and contributions}
\subsubsection{Objectives}
The main objective of this thesis is being able to detect face spoofing attacks made on images with Deep Learning. These images are taken from genuine users (real users) and from people trying to impersonate (attacks) with printed images, masks or images shown on devices such as smartphones or tablets.\\

the second objective is obtaining basic theoretical Deep Learning fundamentals knowledge the same way as learning Theano, the main Deep Learning Python library which is going to be used.\\

The following objective is to build a Convolutional Neural Network architecture which is tailored to this thesis and is, therefore, used to extract the features of the images from the databases.\\

The next objective is working with different classifiers and dimensionality reduction algorithms and selecting the best one from the results obtained.\\

Afterwards, the objective would be selecting the most precise database out of the compared and analyzed results.\\

Finally, the last objective is finding a pattern among the incorrectly classified samples in order to know if physical attributes contribute to the wrong classification.\\

\subsubsection{Contributions}
This thesis contributes with a Convolutional Neural Network architecture and a classifier which optimizes the classification.\\

Also, a study among the different classifiers used and dimensionality reduction algorithms are presented for each database.\\

The next contribution is a discussion among the results obtained in  each database in order to know which database is best, hence, whose samples are the most correctly classified.\\

\section{Thesis structure}
This thesis is constituted by a brief introduction, presented in chapter \ref{ch:introduction}, where the motivation to carry out this project has been exposed and the main objectives of this thesis are described.\\

With the purpose of knowing the advances in the area of this thesis, a literature review is detailed in chapter \ref{ch:review}.\\

Prior to the development of the main part, in chapter \ref{theorics}, a short introduction to neural network and anti-spoofing definitions are presented.\\

In chapter \ref{ch:methodology}, the methodology is explained and the programming language, the databases, classifiers and metrics used in the document are detailed.\\

Experiments that have been carried out are specified in chapter \ref{ch:experiments}, as well as the results obtained.\\

In chapter \ref{ch:discussion}, the results obtained along the experiments are discussed.\\

In the final chapter \ref{ch:concl}, the conclusions obtained from the thesis are explained and the future work is detailed.\\
