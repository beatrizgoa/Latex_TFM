%!TEX root = Memoria_TFM.tex
\minitoc
\mtcskip

\begin{small}
\emph{In this chapter, the reasons that have inspired to effort in developing this work are exposed. The thesis objectives and this document structure are described as well.\\}
\end{small}
\section{Motivation}
Artificial intelligence (AI) is a very researched field which nowadays is having a lot of importance and it is expanding. Because of my interest on AI,  I decided to study an Artificial Vision Master and the most that I would like to investigate is Machine Learning, more exactly, Deep Learning.\\

How bio-inspired algorithms (as Artificial Neural Networks) learn from the input data and classify with a magnificent accuracy the test samples or how Neural Networks are able to create music and pictures impressed me and is a knowledge that I would like to get.\\

Human body as an identification key is fast, private identification and recognition systems. Biometrics is a technology that is currently being implemented in society and its presence in being increasing. At the same time as biometric systems increase its potential, attacks are more powerful. From the acquired knowledge along the Master, I would like to expand them to be able to contribute preventing attacks or improving security of biometrics systems. Applied biometrics has caught my interest. \\

Connect this two technologies, biometrics and Deep learning, is an opportunity to deepen both interesting fields. The Anti-spoofing with CNN researched works has started a few years ago. Developing this thesis could contribute to the investigation.\\

FRAV group are currently working on \textit{ABC4EU European Project}, where face biometric is utilized. Learning from a project that is currently being developed with technology that interested myself, is a generous opportunity. And developing a thesis with them and with biometrics and Deep Learning is an excellent chance that have encouraged me.\\

\section{Objectives and contributions}
The principal objective of this thesis is being able to detect face spoofing attacks made on images with Deep Learning. These images are from genuine users (real users) and images from people trying to impersonate (attacks) with printed images, masks or images shown in devices as smartphones or tablets.\\

Second objective is obtaining Deep Learning basis theoretic knowledge as the same way as learning Theano, the main Python library which is going to be used for

To differentiate between real users and attacks images a Convolutional Neural Network (CNN) has been developed using Python language programming and Theano (a Python library). The secondary objective is learning this artificial intelligence research field that nowadays has a significance importance obtaining a CNN architecture which is able to learn the spoofing features in order to optimize the classification task.\\

Images from different database are feed to the CNN and the output features obtained are the input of the classifiers. Various classifiers has been trained (KNN, logistic regression, Support Vector Machine, etc.) for this purpose. Next goal is obtaining the best classifier that suits better to this anti-spoofing detection problem and if dimensionality reduction algorithms as LDA or PCA are needed.\\

The main contribution of this thesis would be the best combination between a specific CNN architecture and the classifier to classify rightly the most number of samples.\\

Trying to differentiate characteristics of the rightly classified samples and incorrectly classified samples would be interesting. The following objective is identify a pattern (if exists) of incorrectly classified samples.\\

Because of the fact that some databases are going to be used for each experiment, the last presented objective would be  selecting a database whose performance is the best with the classification task and why.\\

\section{Thesis structure}
This thesis is formed by a brief introduction, presented in chapter \ref{ch:introduction}, where the motivation to carry out this project has been exposed and this thesis main objectives are described.\\

With the purpose of knowing the advances in this thesis area, a literature review is detailed in chapter \ref{ch:review}.

Before go ahead to the development of the main part, in chapter \ref{theorics}, a short introduction to neural network is presented as well as anti-spoofing definitions.

In chapter \ref{ch:methodology}, the methodology is presented. In this chapter the language programming, the databases, classifiers and metrics used in the document are detailed.\\

Experiments that have been realized are specified in chapter \ref{ch:experiments}.\\

In chapter \ref{ch:results}, the results obtained in previous chapter \ref{ch:methodology} are presented and discussed.\\

In the last chapter \ref{ch:concl}, the conclusions obtained along the thesis are explained and the future work as well\\
